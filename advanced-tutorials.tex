\section{Advanced Tutorials}

This section outlines some of the more advanced features of cylc along with
links to tutorials on these features on the rose documentation website.


\subsection{Advanced Cycling}
\label{advanced-cycling}

So far we have defined cycling using graph section headings of the form
\lstinline{[[[TXX]]]} as in the following example:

\begin{lstlisting}
[scheduling]
    initial cycle point = 2000-01-01T00
    [[dependencies]]
        [[[T00]]]  # Run every day at 00:00
            graph = foo => bar
\end{lstlisting}

This is the simplest form of graph section heading, more powerful forms exist
to help define complex recursions.

\paragraph*{Examples} $ $

\begin{tabular}{l l}
\begin{lstlisting}
[[[ T06, T1845 ]]]
\end{lstlisting}
& Run every day at 06:00 and 18:45\\
\begin{lstlisting}
[[[ 01T00 ]]]
\end{lstlisting}
& Run every month on the first of the month \\
\begin{lstlisting}
[[[ PT15M ]]]
\end{lstlisting}
& Run every 5 minutes \\
\begin{lstlisting}
[[[ +P5D/P1M ]]]
\end{lstlisting}
& Run every month, starting 5 days after the initial cycle point \\
\begin{lstlisting}
[[[ R3/T06 ]]]
\end{lstlisting}
& Run three times, once every day at 06:00
\\
\end{tabular}

\paragraph*{Tutorial}

TODO: Link to cylc documentation once next version has gone live


\subsection{Advanced Dependencies}

The \& symbol can be used to condense multiple graph lines, for instance the
following example:

\begin{lstlisting}
graph = """
    foo => bar
    foo => baz
"""
\end{lstlisting}

... can be condensed to \lstinline{graph = foo => bar & baz}.
Graph lines can also be written with the \textbar \, symbol meaning or, for
example in the following example \lstinline{baz} will run as soon as either
\lstinline{foo} or \lstinline{bar} succeed.

\begin{lstlisting}
graph = foo | bar => baz
\end{lstlisting}

Up until now we have written dependencies in the form \lstinline{foo => bar}
which means \lstinline{bar} will trigger (run) as soon as \lstinline{foo}
succeeds. It is possible to trigger tasks off of other states e.g:

\begin{tabular}{ll}
\lstinline[mathescape]=foo:fail $=$> bar= &
\lstinline=bar= triggers if \lstinline=foo= fails \\
\lstinline[mathescape]=foo:submit $=$> bar= &
\lstinline=bar= triggers once \lstinline=foo= has submitted \\
\lstinline[mathescape]=foo:start $=$> bar= &
\lstinline=bar= triggers once \lstinline=foo= starts executing \\
\lstinline[mathescape]=foo:finish $=$> bar= &
\lstinline=bar= triggers once \lstinline=foo= succeeds or fails \\
\end{tabular}


\subsection{Families}
\label{Families}

Often a suite will contain a collection of similar tasks. With cylc these
tasks can be grouped together for convenience.

\paragraph*{Example}
In the following example the
task \lstinline{hello_eris} inherits the \lstinline{script} setting and
\lstinline{IS_WORLD_A_PLANET} environment variable from the family
\lstinline{HELLO_FAIMILY}. The task \lstinline{hello_pluto} on the other hand
inherits the \lstinline{script} setting but overrides the
\lstinline{IS_WORLD_A_PLANET} environment variable.

\begin{lstlisting}
[scheduling]
    [[dependencies]]
        graph = HELLO_FAMILY  # Runs all tasks that inherit from HELLO_FAMILY.
[runtime]
    [[HELLO_FAMILY]]  # A family.
        script = echo $IS_WORLD_A_PLANET
        [[[environment]]]
            IS_WORLD_A_PLANET = false # Environment variable shared with tasks
                                      # that inherit from this family.
    [[hello_eris]]
        inherit = HELLO_FAMILY
    [[hello_pluto]]
        inherit = HELLO_FAMILY
        [[[environment]]]
            IS_WORLD_A_PLANET = true  # Overrides the inherited environment
                                      # variable.
\end{lstlisting}

\paragraph*{Tutorial}
\url{http://metomi.github.io/rose/doc/rose-rug-advanced-tutorials-family-trigs.html}.


\subsection{Retries}
Sometimes tasks fail, you can tell cylc to automatically retry failed tasks.

\paragraph*{Example} $ $

\begin{lstlisting}
[runtime]
    [[task]]
        retry delays = 5*PT10S  # Retry up to 5 times waiting 10 seconds before
                                # each retry.
\end{lstlisting}

\paragraph*{Tutorial}
\url{http://metomi.github.io/rose/doc/rose-rug-advanced-tutorials-retries.html}
.


\subsection{Jinja2}

With jinja2 you can set variables and use them throughout your suite.rc file.

\paragraph{Example} $ $

\begin{lstlisting}
#!jinja2

[scheduling]
    [[dependencies]]
        graph = """
            
                task_{{task}} => task_{{task + 1}}
            
        """
[runtime]

[[task_{{task}}]]
    script = echo $TASK_NUMBER
    [[[environment]]]
        TASK_NUMBER = {{task}}

\end{lstlisting}

\paragraph*{Tutorial}
\url{http://metomi.github.io/rose/doc/rose-rug-advanced-tutorials-jinja2.html}


\subsection{Cylc Broadcast}

\lstinline[language=bash]{cylc broadcast} is a command line utility that can
be used to change any setting contained within the \lstinline{[runtime]}
section of a suite.rc file whilst the suite is running.

\paragraph*{Example} The following line of bash script will set the remote
host for the task \lstinline{task_name}.

\begin{lstlisting}[language=bash]
$ cylc broadcast <suite_name> -n <task_name> -s [remote]host=<host_name>
\end{lstlisting}

\paragraph*{Tutorial}
\url{http://metomi.github.io/rose/doc/rose-rug-advanced-tutorials-broadcast.html}


\subsection{Suicide Triggers}

Suicide triggers can be used to remove tasks from a suite's graph during
runtime.

\paragraph*{Example}

The task \lstinline{recover_from_failure} will be removed from the graph if
the task \lstinline{task} succeeds.

\begin{lstlisting}
[scheduling]
    [[dependencies]]
        graph = """
            task:fail => recover_from_failure  # Run recover_from_failure if
                                               # task fails.
            task => !recover_from_failure      # Don't run if task succeeds.
        """
\end{lstlisting}

\paragraph*{Tutorial}
\url{http://metomi.github.io/rose/doc/rose-rug-advanced-tutorials-suicide.html}


\subsection{Queues}

Queues can be used to limit the number of certain tasks that are submitted or
run at any given time.

\paragraph*{Example}

In this example all \lstinline{foo}, \lstinline{bar} and \lstinline{baz} tasks
will go to the \lstinline{task_queue} queue. This queue will only allow two
tasks to run at a time.

\begin{lstlisting}
[scheduling]
    [[queues]]
        [[[task_queue]]]
            limit = 2
            members = foo, bar, baz
\end{lstlisting}

\paragraph*{Tutorial}
\url{http://metomi.github.io/rose/doc/rose-rug-advanced-tutorials-queues.html}


\subsection{Clock Triggered Tasks}
\label{Clock Triggered Tasks}

Sometimes tasks should wait until a certain time before running, in cylc this
is possible using ''clock triggering``.

% TODO - describe as "real date-time dependencies"?

- see \ref{Clock Triggered Tasks}) 
\paragraph*{Example} $ $

\begin{lstlisting}
[scheduling]
    [[dependencies]]
        [[[T00]]]
            graph = daily_task
    [[special tasks]]
        clock-triggered = daily_task(PT0H)  # Run daily_task with 0 hours
                                            # offset from the cycle point.
\end{lstlisting}

\paragraph*{Tutorial}
\url{http://metomi.github.io/rose/doc/rose-rug-advanced-tutorials-clock-triggered.html}.


\subsection{Remote Hosts}

Cylc can run tasks on different machines.

\paragraph*{Example} $ $

\begin{lstlisting}
[runtime]
    [[task]]
        [[[remote]]]
            host = host_name
\end{lstlisting}

\paragraph*{Tutorial}
\url{http://metomi.github.io/rose/doc/rose-rug-advanced-tutorials-remotes.html}
