
\section{Suite Writing Tutorial}
\label{Suite Writing Tutorial}

\subsection{Introduction}

This tutorial walks you through creating a non-trivial suite from scratch, using a simple example. Usually, users will create suites from existing standard template suites.

\subsection{Starting Out}

This example supposes:
\begin{itemize}
    \item We're on a sailing ship, making a passage
    \item We're navigating using a Fortran program
\end{itemize}

Our Fortran program reads in some position data and then pretends to calculate a new one based on a compass direction and a 5 knot speed for a given period of time.

If you are reading this tutorial via the 'metomi VM' then the Fortran code is located at {\em \$HOME/tutorial/suites/navigation/src/dead\_reckoning.f90}. Otherwise, the code is given in Appendix~\ref{Appendix Suite Writing Tutorial}.

We need to analyse our program to figure out what the dependencies are.

Have a quick look through the code and look for:
\begin{itemize}
    \item what files or environment it needs to run
    \item what it produces
    \item when it might need to run
\end{itemize}

\subsection{Inputs}

The inputs to this Fortran code are:
\begin{itemize}
    \item Two environment variables, {\em TIME\_INTERVAL\_HRS} and {\em POSITION\_FILEPATH}
    \item An input (and output) file located at {\em \$POSITION\_FILEPATH} that stores the latitude and longitude.
\end{itemize}

When we run the compiled program, we'll need all these inputs to be present.

We want to run this program every 3 hours, on the hour - so there is a repeated dependency on the time or cycle.

We also want to build the program to begin with, so we need a compilation task that runs at the start.

\subsection{Suite Creation}

If you are running this tutorial on the cylc VM \footnote{
    If and only if you are {\em not} running on the VM:
\begin{itemize}
    \item create a new directory called {\em navigation}) somewhere in your homespace, which will be the top-level suite directory.
    \item inside that directory, create a subdirectory called {\em src}
    \item copy the Fortran code into a file called {\em src/dead\_reckoning.f90}
\end{itemize}
} , the raw directory structure of the suite is set up for you. Change directory to {\em \$HOME/tutorial/suites/navigation/}. This is our suite top-level directory.

\subsubsection{suite.rc file}

We need a {\em suite.rc} file. Create a suite.rc file within your top-level suite directory with the following contents:

\lstset{language=suiterc}
\begin{lstlisting}[columns=fullflexible]
[cylc]
    UTC mode = True # Ignore DST
[scheduling]
    initial cycle point = 20160601T00Z  # 1 June 2016
    final cycle point = 20160603T00Z  # 3 June 2016
    [[dependencies]]
        [[[R1]]]  # Run once at the start of the suite.
            graph = compile_navigate
[runtime]
    [[root]]
        pre-script = sleep 5  # Slow down tasks for visualization.
    [[compile_navigate]]
        script = """
            gfortran $CYLC_SUITE_DEF_PATH/src/dead_reckoning.f90 \
                -o $CYLC_SUITE_SHARE_DIR/dead_reckoning.exe
        """
\end{lstlisting}

    We've set the suite to run from midnight, 1 June 2016 to midnight, 3 June 2016. This is just a model time or label. It isn't synchronised with real clock time.

    We've set the \lstinline{compile_navigate} task to run once at the initial cycle point (i.e. midnight, 1 June 2016). This will compile the source code and produce an executable in a directory {\em \$HOME/cylc-run/navigation/share/}.

    All our tasks inherit from \lstinline{root} and will run \lstinline{sleep 5} from the \lstinline{root} {\em pre-script} setting.

\subsubsection{Initial Run}

    We can test that this works as it is.
    
    Create a blank file called {\em rose-suite.conf} in the top-level suite directory.

    Now you can run \lstinline{rose suite-run}\footnote {We have already seen how we can do the same thing with cylc, and that cylc will absorb the suite-run functionality in the near future.}.
    
    The \lstinline{compile_navigate} task should succeed, and the suite will shut down automatically. After that, there should be a newly created {\em dead\_reckoning.exe} file under {\em \$HOME/cylc-run/navigation/share/}. Have a look in that directory.

\subsubsection{Adding a task}

    Let's now add our navigation task to run our executable. Modify the {\em suite.rc} to have a dependencies section that looks like this:
\lstset{language=suiterc}
\begin{lstlisting}[columns=fullflexible]
    [[dependencies]]
        [[[R1]]]  # Run once at the start of the suite.
            graph = compile_navigate => navigate
        [[[PT3H]]]  # Run every 3 hours (ISO 8601 date-time syntax).
            graph = navigate[-PT3H] => navigate
\end{lstlisting}

Here, we've made \lstinline{navigate} run every 3 hours, with each instance waiting for the previous one to finish.

We've configured the dependency between \lstinline{compile_navigate} and \lstinline{navigate}.

We also need to tell the \lstinline{navigate} task what to run! Add a new entry under the runtime section:

\lstset{language=suiterc}
\begin{lstlisting}[columns=fullflexible]
    [[navigate]]
        script = $CYLC_SUITE_SHARE_DIR/dead_reckoning.exe
        [[[environment]]]
            POSITION_FILEPATH = $CYLC_SUITE_SHARE_PATH/position
            TIME_INTERVAL_HRS = 3
\end{lstlisting}

This instructs cylc to run the executable made by \lstinline{compile_navigate} and sets up the necessary environment variables.

However, we need to add something that makes our {\em \$POSITION\_FILEPATH} position file in the first place.

\subsubsection{Adding another initial-only task}

We'll add a task called \lstinline{write_start_position}. Add it as a run-once ({\em R1}) task by replacing the graph for the \lstinline{[[[R1]]]} section with:

\lstset{language=suiterc}
\begin{lstlisting}[columns=fullflexible]
        [[[R1]]]  # Run once at the start of the suite.
            graph = """
                compile_navigate => navigate
                write_start_position => navigate
            """
\end{lstlisting}

Finally, add these lines at the end of the suite.rc file:

\lstset{language=suiterc}
\begin{lstlisting}[columns=fullflexible]
    [[write_start_position]]
        script = echo '50.0 -3.0' >$CYLC_SUITE_SHARE_PATH/position
\end{lstlisting}

This initialises our location for the \lstinline{navigate} task via a file (pointed to via {\em \$POSITION\_FILEPATH}). Our start coordinates are 50.0 north, 3.0 west, which is in the English Channel (a.k.a. La Manche, Canal da Mancha, etc). You can change these to another location if you like.

\subsection{Checking the suite}

Your suite should now look something like this:

\lstset{language=suiterc}
\begin{lstlisting}[columns=fullflexible]
[cylc]
    UTC mode = True # Ignore DST
[scheduling]
    initial cycle point = 20160601T00Z  # 1 June 2016
    final cycle point = 20160603T00Z  # 3 June 2016
    [[dependencies]]
        [[[R1]]]  # Run once at the start of the suite.
            graph = """
                compile_navigate => navigate
                write_start_position => navigate
            """
        [[[PT3H]]]  # Run every 3 hours (ISO 8601 date-time syntax).
            graph = navigate[-PT3H] => navigate
[runtime]
    [[root]]
        pre-script = sleep 5  # Slow down tasks for visualization.
    [[compile_navigate]]
        script = """
            gfortran $CYLC_SUITE_DEF_PATH/src/dead_reckoning.f90 \
                -o $CYLC_SUITE_SHARE_DIR/dead_reckoning.exe
        """
    [[navigate]]
        script = $CYLC_SUITE_SHARE_DIR/dead_reckoning.exe
        [[[environment]]]
            POSITION_FILEPATH = $CYLC_SUITE_SHARE_PATH/position
            TIME_INTERVAL_HRS = 3
    [[write_start_position]]
        script = echo '50.0 -3.0' >$CYLC_SUITE_SHARE_PATH/position
\end{lstlisting}

\subsection{Running our Suite}

Run:

\begin{lstlisting}[mathescape, language=bash]
$\$$ rose suite-run
\end{lstlisting}

If everything has been set up successfully, after running that command, \lstinline{cylc gui} will launch with your running suite.

\subsection{Finished Output}

You can look at the finished output by running:

\begin{lstlisting}[mathescape, language=bash]
$\$$ rose suite-log
\end{lstlisting}

The position will be written in the out file for each navigate task.

\subsubsection{Nicer Output}

If you want a quick and easy way of visualising the output, try replacing the line:

\lstset{language=Fortran}
\begin{lstlisting}[columns=fullflexible]
PRINT*, "New position, me hearties:",new_lat," ",new_long
\end{lstlisting}

in {\em dead\_reckoning.f90} with

\lstset{language=Fortran}
\begin{lstlisting}[columns=fullflexible]
lat = (180.0/pi) * lat
long = (180.0/pi) * long
CALL get_environment_variable("CYLC_TASK_LOG_ROOT",value=task_log_root,status=code)
OPEN(1,file=TRIM(task_log_root) // '-map.html',action='write')
WRITE(1,'(A, F7.4, A, F7.4, A, F7.4, A, F7.4, A, F7.4, A, F7.4, A)') &
      "<img src='https://maps.googleapis.com/maps/api/staticmap?center=",&
      lat,",",long,"&zoom=7&size=600x300&markers=color:blue|label:A|",&
      lat,",",long,"&markers=color:red|label:B|",new_lat,",",new_long,&
      "'/>"
CLOSE(1)
\end{lstlisting}

This will produce a {\em job-map.html} file per job which you can click through to in \lstinline{Rose Bush}.

\subsection{More Cycling}

Our suite has a start cycle point and a single cycling period of 3 hours, but we could have other cycle definitions in the same suite. Let's add a task called \lstinline{take_sun_sight} that runs at 12:00 each day. This will correct our latitude.

Add these lines below \lstinline{[[dependencies]]} in your suite.rc file:

\lstset{language=suiterc}
\begin{lstlisting}[columns=fullflexible]
        [[[T12]]]
            graph = navigate => take_sun_sight
\end{lstlisting}

This will run after the \lstinline{navigate} task from that cycle time finishes.

In order to make the \lstinline{navigate} task wait for our new \lstinline{take_sun_sight} task, we'll need to add some extra configuration for the 15 hour cycle - add the following lines in the same way as you did for \lstinline{[[[T12]]]}:

\lstset{language=suiterc}
\begin{lstlisting}[columns=fullflexible]
        [[[T15]]]
            graph = take_sun_sight[-PT3H] => navigate
\end{lstlisting}


\subsection{Adding a Script to the Suite}

You can put scripts in the {\em bin/} directory of a suite, and they will be added to cylc's path.

Change directory to the root of your suite, and create a {\em bin/} directory.

Create an empty file in the {\em bin/} directory called {\em bin/sun\_sight}.

Open this file with a text editor. Paste the following text into the {\em sun\_sight} file:

\lstset{language=Python}
\begin{lstlisting}[columns=fullflexible]
#!/usr/bin/env python

import random
import sys


if __name__ == "__main__":
    random.seed()
    with open(sys.argv[1], "r") as f:
        (lat, long) = f.read().split()
    lat = float(lat) + random.uniform(-0.05, 0.05)
    print "Yarr! Our corrected position be {0}, {1}".format(lat, long)
    with open(sys.argv[1], "w") as f:
        f.write("{0} {1}\n".format(lat, long))
\end{lstlisting}

Save the file and make sure the indentation is correct. Make it executable by running:

\begin{lstlisting}[mathescape, language=bash]
$\$$ chmod +x bin/sun_sight
\end{lstlisting}

We need to reference this script explicitly in the {\em suite.rc} for our \lstinline{take_sun_sight} task - append these lines to the {\em suite.rc} file:

\lstset{language=suiterc}
\begin{lstlisting}[columns=fullflexible]
    [[take_sun_sight]]
        script = sun_sight $CYLC_SUITE_SHARE_PATH/position
\end{lstlisting}

\subsection{Results}

Run the suite by invoking:
\begin{lstlisting}[mathescape, language=bash]
$\$$ rose suite-run
\end{lstlisting}

Our extra task should run at 20160601T1200Z and 20160602T1200Z. If it does, congratulations!

You can also check the graphing by running:

\begin{lstlisting}[mathescape, language=bash]
$\$$ cylc graph navigation 20160601T0000Z 20160603T0000Z
\end{lstlisting}

\subsection{Rose (optional - feel free to skip)}
\label{Rose}

Apart from the \lstinline{rose suite-run} functionality (which will soon be absorbed into cylc), Rose can help us manage configuration for our suites and tasks.

Let's pull out our initial starting longitude and latitude into top-level {\em rose-suite.conf} Jinja2 template inputs. Jinja2 allows us to effectively script the {\em suite.rc} file when we need to.

Add this line to the top of the suite.rc file:

\lstset{language=suiterc}
\begin{lstlisting}[columns=fullflexible]
#!jinja2
\end{lstlisting}

and replace:

\lstset{language=suiterc}
\begin{lstlisting}[columns=fullflexible]
    [[write_start_position]]
        script = echo '50.0 -3.0' > $CYLC_SUITE_SHARE_PATH/position
\end{lstlisting}

with:

\lstset{language=suiterc}
\begin{lstlisting}[columns=fullflexible]
    [[write_start_position]]
        script = echo '{{START_LATITUDE}} {{START_LONGITUDE}}' > $CYLC_SUITE_SHARE_PATH/position
\end{lstlisting}

We can now pull out \lstinline{START_LATITUDE} and \lstinline{START_LONGITUDE} into top-level {\em rose-suite.conf} inputs. Change the {\em rose-suite.conf} file to read:

\lstset{language=suiterc}
\begin{lstlisting}[columns=fullflexible]
[jinja2:suite.rc]
START_LATITUDE=50.0
START_LONGITUDE=-3.0
\end{lstlisting}

When the suite is run with \lstinline{rose suite-run}, these variables will be inserted into the \lstinline{write_start_position} task's script and it will run as before.

Rose has a concept of metadata for inputs like these, which help check their validity and improve the presentation in the \lstinline{rose edit} configuration editor GUI. We can take a shortcut to fill out some of that metadata.

In the top-level directory of your suite, run:

\begin{lstlisting}[mathescape, language=bash]
$\$$ rose metadata-gen --auto-type
\end{lstlisting}

It will create a subdirectory called {\em meta} with a file called {\em rose-meta.conf} inside. Open that file and edit it to add some description and allowed value ranges:

\lstset{language=suiterc}
\begin{lstlisting}[columns=fullflexible]
[jinja2:suite.rc]
title=Starting Position

[jinja2:suite.rc=START_LATITUDE]
compulsory=true
description=Starting latitude in degrees
range=-90:90
type=real

[jinja2:suite.rc=START_LONGITUDE]
compulsory=true
description=Starting longitude in degrees
range=-180:180
type=real
\end{lstlisting}
 
Now run the command \lstinline{rose edit} - this launches a nice GUI interface to our inputs which will check for errors. Try e.g altering the longitude to 300 degrees. An error should display.

You can run the suite directly through \lstinline{rose edit} via the 'Play' button in the toolbar.

We can also add configuration for our Fortran program via a Rose app configuration. Alter the \lstinline{navigate} task \lstinline{script} in the \lstinline{suite.rc} file to read:
\lstset{language=suiterc}
\begin{lstlisting}[columns=fullflexible]
    [[navigate]]
        script = rose task-run -v
\end{lstlisting}

Create an {\em app/navigate/} subdirectory in your suite:

\begin{lstlisting}[mathescape, language=bash]
$\$$ mkdir -p app/navigate/
\end{lstlisting}

Create a file underneath that directory called {\em rose-app.conf} ({\em app/navigate/rose-app.conf}). Alter that file to read:

\lstset{language=suiterc}
\begin{lstlisting}[columns=fullflexible]
[command]
default=$CYLC_SUITE_SHARE_DIR/dead_reckoning.exe

[env]
POSITION_FILEPATH=$CYLC_SUITE_SHARE_PATH/position
TIME_INTERVAL_HRS=3
\end{lstlisting}

We could now remove the \lstinline{[[[environment]]]} settings for \lstinline{[[navigate]]} in the {\em suite.rc} file.

Create a further {\em meta} subdirectory for some app metadata:

\begin{lstlisting}[mathescape, language=bash]
$\$$ mkdir -p app/navigate/meta
\end{lstlisting}

Inside that new directory, create a {\em rose-meta.conf} file ({\em app/navigate/meta/rose-meta.conf}) that looks like this:

\lstset{language=suiterc}
\begin{lstlisting}[columns=fullflexible]
[env=TIME_INTERVAL_HRS]
range=1:24
type=integer
\end{lstlisting}

If you save these files and then launch \lstinline{rose edit}, you should see configuration appear for the \lstinline{navigate} app. As we have declared \lstinline{TIME_INTERVAL_HRS} to be an integer, it should have a numeric widget. It should display an error if the value is less than 1 or greater than 24.

This is only a limited subset of the functionality that Rose apps and metadata can provide - there is also inter-variable triggering, a Pythonic mini-language, dynamic page assignment, deep Fortran namelist integration, and much more. See the Rose documentation for more details.
