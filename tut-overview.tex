\section{Tutorial Scope}

This is intended to be a one-day hands-on tutorial for new cylc users.  Cylc,
Rose, this document, and the tutorial example suites are installed on a
Linux VM that can be downloaded from:
\begin{itemize}
    \item \url{http://https://github.com/metomi/metomi-vms}
\end{itemize}

Automating large distributed workflows in operational environments can be
complicated, and the full cylc user guide may look intimidating to new users as
a result.  Simplicity and ease of use for smaller workflows has always been a
primary goal of the cylc project, however, and we have endeavored to convey
this in the tutorial. Those who want to know more can refer to supplementary
material provided in the Appendices, and other cylc documentation:

\begin{itemize}
    \item \url{http://cylc.github.io/cylc}
\end{itemize}

\terminology{Cylc \underline{terminology} will be \underline{defined} and
highlighted like this.}

\note{Pointers to additional information will be highlighted like this.}

\begin{shaded*}
Embedded tutorials are in shaded boxes like this.
\end{shaded*}

\subsection{A Note On Rose}

Rose is an open source system for suite (workflow) management, designed for use
with cylc.
\begin{itemize}
     \item \url{http://metomi.github.io/rose}
\end{itemize}
 
It is installed on your VM if want to try it.  Rose functionality
includes:
\begin{itemize}
    \item Suite storage, discovery, and revision control.  This enables sharing
        and collaborative development of suites across the Unified Model
        Consortium.
    \item Generic configuration, and metadata-driven configuration editing, of
        suites and tasks, including large scientific models.
    \item Various utilities for use with cylc suites.
\end{itemize}

The last bullet point includes some programs that will soon be migrated into
the cylc project.  We will be using two of these, pre-migration, in this
tutorial: \lstinline{rose suite-run} - a better way of running cylc suites; and
\lstinline{rose bush} - a sophisticated web-based suite log viewer.
