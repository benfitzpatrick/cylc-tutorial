\section{Tutorial Scope}

This is intended to be a one-day hands-on tutorial for new cylc users.

Automating large distributed workflows of cycling tasks in research and
operational environments can be a complex business, and consequently the full
cylc user guide may look somewhat intimidating to new users.  Ease of use
has always been a primary goal of the project, however.  We have endeavored to
keep to the basics and convey this simplicity in the tutorial. Those who want
to know more can refer to supplementary material provided in the Appendices,
and other cylc documentation:

\begin{itemize}
    \item \url{http://cylc.github.io/cylc}
\end{itemize}

Cylc terminology will be highlighted and defined as needed, like this:

\terminology{A \underline{task} represents a \underline{job} (a
script or program) that runs on a computer.}

\subsection{A Note On Rose}

Rose is an open source system for suite (workflow) management, designed for use
with cylc.
\begin{itemize}
     \item \url{http://metomi.github.io/rose}
\end{itemize}
 
It is installed on your metomi VM if want to try it.  Rose functionality
includes:
\begin{itemize}
    \item Suite storage, discovery, and revision control.  This enables sharing
        and collaborative development of suites across the Unified Model
        Consortium.
    \item Generic configuration, and metadata-driven configuration editing, of
        suites and tasks, including large scientific models.
    \item Various utilities for use with cylc suites.
\end{itemize}

The last bullet point includes some programs that will soon be migrated into
the cylc project.  We will be using two of these, pre-migration, in this
tutorial: \lstinline{rose suite-run} - a better way of running cylc suites; and
\lstinline{rose bush} - a sophisticated web-based suite log viewer.
